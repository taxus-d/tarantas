\documentclass{article}
\usepackage[utf8]{inputenc}
\usepackage[T2A]{fontenc}
\usepackage{trkvprefix}

\def\prettylist{$=\!$[\k$\to$\v]$\!=\!$}
\newPairParser{\getPrettyPair}{\prettylist}
\newPairsParser{\getPrettyPairs}{\prettylist}
\def\testlist{}
% coerced etoolbox to produce special list parser for trkv
% no braces problem now
\def\do#1{\listadd\testlist{#1}}
\dokvlist{{formulas={\it none}, math, etc}
=
$b$  ,                   {  c .= lf}
              =
{d<<=>>k l}, \TeX    =\TeX}
\makeatletter
\newPairParser{\storeto}[2]{
  \kT=\ex@f\ex@f\ex@f{\ex@f\detokenize\ex@f{\the\kT}}%
  \csedef{#1\k}{\v}%
  \cssmuggle{#1\k}%
}
\makeatother
\begin{document}
keys: \forlistloop{$|$\getKey}{\testlist}$|$ \par
values: \forlistloop{$|$\getVal}{\testlist}$|$ \par
pairs: \forlistloop{\getPrettyPair}{\testlist} \par
\getPrettyPairs{guk=1,hek = \def\hek#1{{\bf#1}}\hek{kek}}\par
\storePair{/kek/color/.ini = deep red with bits of blue}
\fboxsep=1pt
\fbox{\getByKey{/kek/color/.ini}}\par
\storePairs{sigma = $\sigma$, exit = no way, coding in = \TeX}
\fbox{\getByKeys{sigma,exit, coding in}} \par
\fbox{\getKey{can i haz = #1#2}}


\storePair{a func! = its body! like: #1#2}%
\edef\x{\def\noexpand\ttt##1##2{\getByKey{a func!}}}\x
\fbox{\ttt {$\int$} {{{$\sum$}}}}
\relax

% And a stressful one
\getPrettyPairs{aa,bb, =aaa, \# =jjj}

% multitoken separator is supported, to
\setPairSep{->}
\setKvListSep{;}
\getPrettyPairs{ with `=' sign->can store them!; and this-> too}

\storePairDef{sigma-> $\sigma\!\sigma$}
\storePairsDef{sigme -> $\sigma\!e$; sigmo -> $\sigma\!o$}
\setPairSep{:}
\getPairsByKey{sigme, sigma, sigmo}

\storePair{
 and
 even
 empty
 one: }  
\par

\ProvidesFile{trkv.def}[2017/07/10
  Key=Val parser defaults for `tarantas' bundle]

\setPairSep{=}
\setListSep{,}
\setStorePrefix{trkv@keys@}

% vim:ft=tex


\newPairsParser{\getKeysList}{
  \expandafter\listgadd\expandafter\keylist \expandafter{\romannumeral-`X\k}
}

\def\keylist{}
\getPrettyPairs{$\sin'$=$\cos$, $\tan'$=$\frac{1}{\cos^2}$}
\getKeysList{$\sin'$=$\cos$, $\tan'$ = $\frac{1}{\cos^2}$}

\storePairDef{note=\footnote}
\edef\foo{
\noexpand\toks0={\getByKey{note}}
}want to mention that\the\toks0{and this}
% \def \defhelper#1{
%   \getByKey{notes/#1}
% }
% \expandafter \defhelper \getByKey{note}

\storeto{eleven crabs}{next crab=taxus}
\rawgetByKey{eleven crabs}{next crab}


\end{document}
% TEX program = TEXINPUTS=../..//:: 
