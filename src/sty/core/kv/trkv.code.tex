% BEWARE
% I wrote this code at one night in spring 2018.
% Only God knows now how it works.

%% useful condition
\newif \iftrkv@expand
%% a few common variables
\newtoks \trkv@pair@kT
\newtoks \trkv@pair@vT

%% naming hint
% pairparser -- entity that assepts a whole pair  (#1<-{a=b})
% kvparser   -- splits kv pair in its args, low-level

%% name-generating macros
\def \trkv@pairparser@cs@kvparse#1{\csname trkv@pp\ex@f\@gobble\string#1@kvparse\endcsname}
\def \trkv@pairparser@cs@singular#1{\csname trkv@pp\ex@f\@gobble\string#1@singular\endcsname}
\def \trkv@pairparser@cs@list#1{\csname trkv@pp\ex@f\@gobble\string#1@listparser\endcsname}
\def \trkv@pairparser@cs@tokspreproc#1{\csname trkv@pp\ex@f\@gobble\string#1@tokspreproc\endcsname}
\def \trkv@pairparser@code#1{\csname trkv@pp\ex@f\@gobble\string#1@code\endcsname}


%% real parsing

% new parser, front end
\def \trkv@pairparser@new{%
  \@ifnextchar[%]
    {\trkv@pairparser@declare}%
    {\trkv@pairparser@declare[{=}{,}]}%
}
\def \trkv@pairparser@declare[#1#2]#3{%
  \@ifnextchar[%]
    {\trkv@pairparser@def[{#1}{#2}]{#3}}%
    {\trkv@pairparser@def[{#1}{#2}]{#3}[1]}% sole arg is default
}


% [{<tl>/pair-sep/} {<tl>/list-sep/}] {<cs>/parser name/} [<num>/number of args/] {<tl>/code/}
%% preproc is an action on token list \toks@
\def \trkv@pairparser@def[#1#2]#3[#4]#5{%
  \iftrkv@debug\trt@arg@debug{pp([{#1}{#2}]#3[#4]#5)}\fi%
% kvparser, lowlevel one
  \trkv@kvparser@def{#1}{#3}%
  \iftrkv@debug\ex@f\ex@f\ex@f \show \trkv@pairparser@cs@kvparse{#3}\fi%
%
% singular pair parser, parses a single pair
  \trkv@singularpairparser@def{#1}{#3}%
  \iftrkv@debug\ex@f\ex@f\ex@f \show \trkv@pairparser@cs@singular{#3}\fi%
%
% list parser
  \ex@f\ex@f\ex@f\trkv@arglistparser@def\ex@f\ex@f\ex@f%
    {\trkv@pairparser@cs@list{#3}}%
    {#2}%
  %
  \iftrkv@debug\ex@f\ex@f\ex@f \show\trkv@pairparser@cs@list{#3}\fi%
%
% final user command
  \iftrkv@expand%
    \ex@f\ex@f\ex@f\def\trkv@pairparser@cs@tokspreproc{#3}{\ex@f\trt@macro@fullexpand\ex@f{\the\toks@}->T{\toks@}}%
  \else%
    \ex@f\ex@f\ex@f\def\trkv@pairparser@cs@tokspreproc{#3}{\relax}%
  \fi%
    \trkv@pairparser@defcommand{#3}{#4}{#5}{\relax}%
  \iftrkv@debug\show #3\fi%
}


% {<cs>/command name/} {<num>/number of args/} {/code/} 
%% preproc is an action on token list \toks@
\def \trkv@pairparser@defcommand#1#2#3{
  \edef \trt@exphandler{%
    \unexpanded{\newcommand{#1}[#2]}{%
      % we are embedding it here to allow proper argument substitution
      %+ in parser code
      \def \expandtwice{\trkv@pairparser@code{#1}{#3}}%
      \noexpand\toks@={#####2}%
      \expandtwice{\trkv@pairparser@cs@tokspreproc{#1}}%
      \unexpanded{\ex@f\trkv@pairparser@run\ex@f{\the\toks@}}%
      \expandtwice{\trkv@pairparser@cs@list{#1}}%
      \expandtwice{\trkv@pairparser@cs@singular{#1}}%
    }%
  }\trt@exphandler%
}


% {<tl>/args/} {<cs>/list parser/} {<cs>/singular parser/}
\def \trkv@pairparser@run#1#2#3{#2{#3}{#1}}


% {<tl>/pair-sep/} {<cs>/parser name/}
%% aaa=bbb=ccc\@nil. Called as <pair>==\@nil
%% <pair> is a=b ==> #1 <- 'a', #2 <- 'b', #3 <- '='
%% <pair> is a   ==> #1 <- 'a', #2 <- '' , #3 <- ''
\def \trkv@kvparser@def#1#2{%
  \ex@f\ex@f\ex@f \def\trkv@pairparser@cs@kvparse{#2}##1#1##2#1##3\@nil{%
    \trt@str@trim{##1}->T{\kT}%
    \trt@str@trim{##2}->T{\vT}%
  }%
}% small slowdown \iftrkv@debug \trt@arg@debug{kv:[##1]=[##2]}\fi%


% {<tl>/pair-sep/} {<cs>/parser name/} {<tl>/code/}
\def \trkv@singularpairparser@def#1#2{%
  \ex@f\ex@f\ex@f \def\trkv@pairparser@cs@singular{#2}##1{%
    \if\relax\detokenize{##1}\noexpand\relax%
      \PackageError{trkv}{\string\trkv@pair@parse Nothing to parse}%
        {Provide a nonempty string. Or invent a workaround}%
    \else%
    \begingroup%
      % holders of key and value
      \let\kT\trkv@pair@kT%
      \let\vT\trkv@pair@vT%
      \trkv@pairparser@cs@kvparse{#2}##1#1#1\@nil%
      \def\k{\the\kT}%
      \def\v{\the\vT}%
      \trkv@pairparser@code{#2}% <-- user code
    \endgroup%
    \fi%
  }%
}

% vim: ft=plaintex
